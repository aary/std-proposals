% ********************************* HEADERS ***********************************
\documentclass{article}
\usepackage[top=.75in, bottom=.75in, left=.50in,right=.50in]{geometry}
\usepackage{fancyhdr}
\usepackage{titling}
\pagestyle{fancy}
\lhead{Remove return type deduction in std::apply}
\rhead{\thepage}
\usepackage{color}
\usepackage{graphicx}
\usepackage{listings}
\lstset{escapechar={|}}
\usepackage[utf8]{inputenc}
\usepackage[spanish]{babel}
\usepackage{csquotes}
\lstset{
  % language=C++,
  showstringspaces=false,
  basicstyle={\small\ttfamily},
  numberstyle=\tiny\color{gray},
  keywordstyle=\color{blue},
  commentstyle=\color{dkgreen},
  stringstyle=\color{dkgreen},
}
\usepackage[colorlinks,urlcolor={blue}]{hyperref}
%\setlength{\parskip}{1em}
\usepackage{parskip}
\usepackage{indentfirst}
\setlength{\droptitle}{-4em}
\usepackage{soul}
\usepackage{color}
\DeclareRobustCommand{\hlgreen}[1]{{\sethlcolor{green}\hl{#1}}}
\newcommand*\justify{%
  \fontdimen2\font=0.4em% interword space
  \fontdimen3\font=0.2em% interword stretch
  \fontdimen4\font=0.1em% interword shrink
  \fontdimen7\font=0.1em% extra space
  \hyphenchar\font=`\-% allowing hyphenation
}
\usepackage{xcolor}
\definecolor{darkgreen}{rgb}{0.0, 0.5, 0.0}
\lstdefinestyle{base}{
  basicstyle={\color{darkgreen}\small\ttfamily},
  showstringspaces=false,
  numberstyle=\tiny\color{gray},
  keywordstyle=\color{blue},
  commentstyle=\color{dkgreen},
  stringstyle=\color{dkgreen},
}
% \usepackage{concmath}
% \usepackage[T1]{fontenc}
% ********************************* HEADERS ***********************************

\begin{document}
\title{\textbf{Remove return type deduction in \texttt{std::apply}}}
\author{Aaryaman Sagar (aary@instagram.com)}
\date{October 7, 2018}
\maketitle

\section{Introduction}

\begin{lstlisting}
#include <tuple>
#include <iostream>

template <typename VoidT, typename Func, typename... Args>
struct well_formed : std::false_type {};

template <typename Func, typename... Args>
struct well_formed<
    std::void_t<decltype(
        std::apply(std::declval<Func>(), std::declval<Args>()...))>,
    Func,
    Args...> : std::true_type {};

void foo(bool) {}

int main() {
  auto func = []() {};
  auto args = std::make_tuple(1);

  std::ignore = well_formed<void, decltype(func)&, decltype(args)&>{};
}
\end{lstlisting}

The code above should be well formed.  However, since \texttt{std::apply} uses
return type deduction to deduce the return type, we get a hard error as the
substitution is outside the immediate context of the template instantiation.

This paper proposes a new public trait instead of \texttt{decltype(auto)} in
the return type of of \texttt{std::apply}

\section{Impact on the standard}
This proposal is a pure library extension.

\section{\texttt{std::apply\_result}}
\texttt{std::apply\_result} (and the corresponding alias
\texttt{std::apply\_result\_t}) is the proposed trait that should be used in
the return type of \texttt{std::apply}.  With the new declaration being

\begin{lstlisting}
template <class F, class Tuple>
constexpr std::apply_result_t<F, Tuple> apply(F&& f, Tuple&& t);
\end{lstlisting}

This fixes hard errors originating from code that tries to employ commonly-used
SFINAE patterns with \texttt{std::apply} that could have otherwise been well
formed.  And is backwards compatible with well-formed usecases of
\texttt{std::apply}

\section{Implementation}
\texttt{std::apply\_result} can be defined using the existing
\texttt{std::invoke\_result} trait to avoid duplication in implementations

\begin{lstlisting}
// apply_impl is for exposition only
template <class F, class T, std::size_t... I>
constexpr auto apply_impl(F&& f, T&& t, std::index_sequence<I...>) noexcept(
    is_nothrow_invocable<F&&, decltype(std::get<I>(std::declval<T>()))...>{})
    -> invoke_result_t<F&&, decltype(std::get<I>(std::declval<T>()))...> {
  return invoke(std::forward<F>(f), std::get<I>(std::forward<T>(t))...);
}

template <typename F, typename Tuple>
using apply_result_t = decltype(apply_impl(
    std::declval<F>(),
    std::declval<Tuple>(),
    std::make_index_sequence<std::tuple_size_v<std::decay_t<Tuple>>>{}));

template <typename F, typename Tuple, typename = std::void_t<>>
class apply_result {};
template <typename F, typename Tuple>
class apply_result<F, Tuple, std::void_t<std::apply_result_t<F, Tuple>>> {
  using type = std::apply_result_t<F, Tuple>;
};
\end{lstlisting}

\section{Changes to the standard}
\subsection{Section 23.5.2 (\texttt{[[tuple.syn]]})}
\begin{lstlisting}
    // 23.5.3.5, calling a function with a tuple of arguments
    template <class F, class Tuple>
    constexpr |\color{darkgreen}std::apply\_result\_t<F, Tuple>| apply(F&& f, Tuple&& t);
\end{lstlisting}
\begin{lstlisting}[style=base]
    template <class F, class Tuple> class apply_result;
    template <class F, class Tuple> using apply_result_t = (see below);
\end{lstlisting}

\subsection{Section 23.5.3.5 (\texttt{[tuple.apply]} paragraph 1)}
\begin{lstlisting}
    template <class F, class Tuple>
    constexpr |\color{darkgreen}std::apply\_result\_t<F, Tuple>| apply(F&& f, Tuple&& t) |\color{darkgreen}noexcept(see below);|
\end{lstlisting}

Effects: Given the exposition-only function:

\begin{lstlisting}[style=base]
    template <class F, class T, std::size_t... I>
    constexpr auto apply_impl(F&& f, T&& t, std::index_sequence<I...>) noexcept(
        is_nothrow_invocable<F&&, decltype(std::get<I>(std::declval<T>()))...>{})
        -> invoke_result_t<F&&, decltype(std::get<I>(std::declval<T>()))...> {
      return invoke(std::forward<F>(f), std::get<I>(std::forward<T>(t))...);
    }
\end{lstlisting}

Equivalent to:

\begin{lstlisting}[style=base]
  return apply_impl(
      std::forward<F>(f),
      std::forward<Tuple>(t),
      std::make_index_sequence<std::tuple_size<std::decay_t<Tuple>>{}>{});
\end{lstlisting}

\textcolor{darkgreen}{Remarks: The expression inside noexcept is equivalent to:}

\begin{lstlisting}[style=base]
    apply_impl(std::forward<F>(f), std::forward<Tuple>(t),
               make_index_sequence<tuple_size_v<decay_t<Tuple>>>|{});
\end{lstlisting}

\subsection{Section 23.5.3.6 (\texttt{[tuple.helper]})}

\begin{lstlisting}[style=base]
    template <class F, class Tuple> using apply_result_t = (see below)
\end{lstlisting}

\textcolor{darkgreen}{Effects: Given the exposition-only function:}

\begin{lstlisting}[style=base]
    template <class F, class T, std::size_t... I>
    constexpr auto apply_impl(F&& f, T&& t, std::index_sequence<I...>) noexcept(
        is_nothrow_invocable<F&&, decltype(std::get<I>(std::declval<T>()))...>{})
        -> invoke_result_t<F&&, decltype(std::get<I>(std::declval<T>()))...> {
      return invoke(std::forward<F>(f), std::get<I>(std::forward<T>(t))...);
    }
\end{lstlisting}

\textcolor{darkgreen}{Equivalent to:}

\begin{lstlisting}[style=base]
    decltype(apply_impl(std::forward<F>(f), std::forward<Tuple>(t),
             make_index_sequence<tuple_size_v<decay_t<Tuple>>>{});
\end{lstlisting}

\begin{lstlisting}[style=base]
    template <typename F, typename Tuple, typename = void_t<>>
    class apply_result {};
    template <typename F, typename Tuple>
    class apply_result<F, Tuple, void_t<apply_result_t<F, Tuple>>> {
  	    using type = apply_result_t<F, Tuple>;
    };
\end{lstlisting}

\textcolor{darkgreen}{If the expression \texttt{apply\_impl(std::declval<F>(),
std::declval<Tuple>())} is well-formed when treated as an unevaluated operand
(Clause 8), the member typedef \texttt{type} names the type
\\\texttt{decltype(apply\_impl(std::declval<F>(), std::declval<Tuple>()))}.
Otherwise there shall be no member type.  Access checking is performed as if
in a context unrelated to \texttt{F} and \texttt{Tuple}.  Only the validaity
of the immediate context is considered.  [Note: The compilation of the
expression can result in side effects such as the instantiation of class
template specializations and function template specializations, the generation
of implicitly-defined functions and so on.  Such side-effects are not in the
"immediate context" and can result in the program being ill-formed.  - end
note]}

\end{document}
