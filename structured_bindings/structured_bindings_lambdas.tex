% ********************************* HEADERS ***********************************
\documentclass{article}
\usepackage[top=.75in, bottom=.75in, left=.50in,right=.50in]{geometry}
\usepackage{fancyhdr}
\usepackage{titling}
\pagestyle{fancy}
\lhead{Structured bindings with polymorphic lambas}
\rhead{\thepage}
\usepackage{color}
\usepackage{graphicx}
\usepackage{listings}
\lstset{
  language=C++,
  showstringspaces=false,
  basicstyle={\small\ttfamily},
  numberstyle=\tiny\color{gray},
  keywordstyle=\color{blue},
  commentstyle=\color{dkgreen},
  stringstyle=\color{dkgreen},
}
\usepackage[colorlinks,urlcolor={blue}]{hyperref}
%\setlength{\parskip}{1em}
\usepackage{parskip}
\usepackage{indentfirst}
\setlength{\droptitle}{-4em}
\usepackage{soul}
\usepackage{color}
\DeclareRobustCommand{\hlgreen}[1]{{\sethlcolor{green}\hl{#1}}}
\newcommand*\justify{%
  \fontdimen2\font=0.4em% interword space
  \fontdimen3\font=0.2em% interword stretch
  \fontdimen4\font=0.1em% interword shrink
  \fontdimen7\font=0.1em% extra space
  \hyphenchar\font=`\-% allowing hyphenation
}
% \usepackage{concmath}
% \usepackage[T1]{fontenc}
% ********************************* HEADERS ***********************************

\begin{document}
\title{\textbf{Structured bindings with polymorphic lambas}}
\author{Aaryaman Sagar (aary800@gmail.com)}
\date{August 14, 2017}
\maketitle

\section{Introduction}

This paper proposes usage of structured bindings with polymorphic lambdas,
adding them to another place where \texttt{auto} can be used as a declarator

\begin{lstlisting}
std::for_each(map, [](auto [key, value]) {
    cout << key << " " << value << endl;
});
\end{lstlisting}

This would make for nice syntactic sugar to situations such as the above
without having to decompose the tuple-like object manually, similar to how
structured bindings are used in range based for loops

\begin{lstlisting}
for (auto [key, value] : map) {
    cout << key << " " << value << endl;
}
\end{lstlisting}


\section{Motivation}

\subsection{Simplicity and uniformity}
Structured binding initialization can be used almost anywhere \texttt{auto} is
used to initialize a variable (not considering \texttt{auto} deduced return
types), and allowing this to happen in polymorphic lambdas would make code
simpler, easier to read and generalize better

\begin{lstlisting}
std::find_if(range, [](const auto& [key, value]) {
    return examine(key, value);
});
\end{lstlisting}

\subsection{Programmer demand}
There is some programmer demand and uniform agreement on this feature

\begin{enumerate}
    \item \href{https://stackoverflow.com/questions/45541334}{Stack Overflow:
        Can the structured bindings syntax be used in polymorphic lambdas}
    \item \href{https://goo.gl/fRSwNg}{ISO C++ : Structured bindings and
        polymorphic lambdas}
\end{enumerate}

\subsection{Prevalence}
It is not uncommon to execute algorithms on containers that contain a value
type that is either a tuple or a tuple-like decomposable class.  And in such
cases code usually deteriorates to manually unpacking the instance of the
decomposable class for maximum reaadability, for example

\begin{lstlisting}
auto result = std::count_if(map, [](const auto& key_value_pair) {
    const auto& key = key_value_pair.first;
    const auto& value = key_value_pair.second;

    return examine(key, process_key(key), value);
});
\end{lstlisting}

The first two lines in the lambda are just noise and can nicely be replaced
with structured bindings in the function parameter

\begin{lstlisting}
auto result = std::count_if(map, [](const auto& [key, value]) {
    return examine(key, process_key(key), value);
});
\end{lstlisting}


\section{Impact on the standard}
The proposal describes a pure language extension which is a non-breaking
change - code that was previously ill-formed now would have well-defined
semantics.


\section{Interaction with concepts and traits}
\textbf{Definition (\texttt{x-decomposable})} If a type can be decomposed into
a structured binding expression with \texttt{x} bindings, then it is said to
be \texttt{x-decomposable} (reference \texttt{[dcl.struct.bind]} for the exact
requirements).  More specifically, if the following expression is well formed
in the least restrictive member access scope (where privates are accessible)
for a type \texttt{T}
\begin{lstlisting}
auto&& [one, two, three, ... , x] = o;
\end{lstlisting}
Where \texttt{decltype(o)} is \texttt{T}, then the type \texttt{T} is said to
be \texttt{x-decomposable}

This can be made available to the compiler as both a concept and a trait.  The
presence of such a concept makes it easy to define templates in terms of a
type that is \texttt{x-decomposable}.  A trait allows for the same thing but
can be considered more versatile as it also fits well with existing code that
employs value driven template specialization mechanisms and other more
complicated specialization workflows.

It is possible to make a concept or trait that enables us to check if a type
is decomposable into \texttt{x} bindings by virtue of it's interface.  In
particular the presence of an ADL defined or member \texttt{get<>()} function
and the existence of specialized \texttt{std::tuple\_element<>} and
\texttt{std::tuple\_size<>} traits qualifies something to be
\texttt{x-decomposable}.  However, a type can be \texttt{x-decomposable} even
when these are not present (see \texttt{[dcl.struct.bind]}p4)

\texttt{[dcl.struct.bind]}p2 and \texttt{[dcl.struct.bind]}p3 define
decomposability that can be checked by the programmer at compile time
(described above) via a concept or trait.  However
\texttt{[dcl.struct.bind]}p4 describes a method of unpacking that cannot be
enforced purely by the language constructs available as of C++17.  As such a
compiler intrinsic, say \texttt{\_\_is\_decomposable<T, x>} is required.
Given such an intrinsic, defining a trait and concept that check if a type is
\texttt{x-decomposable} on top of that is trivial.  The trait itself can be
used as a backend for the concept, leaving the implementation of the concept
entirely in portable code without the help of compiler intrinsics.

The concept, say \texttt{std::decomposable<x>} accepts a non type template
parameter of type \texttt{std::size\_t} that determines the cardinality of the
structured bindings decomposition.  This concept holds if a type is
\texttt{x-decomposable} (and this will take into consideration the
requirements set forth by \texttt{[dcl.struct.bind]} paragraphs 2, 3 and 4.

The corresponding trait, say \texttt{std::is\_decomposable<T, x>} has value
\texttt{true} if and only if type \texttt{T} is \texttt{x-decomposable}.  The
usual variable template \texttt{std::is\_decomposable\_v<T, x>} should also be
defined.


\section{Impact on overloading and function resolution}
Lambdas do not natively support function overloading, however one can lay out
lambdas in a way that they are overloaded, for example let's assume the
following definition of \texttt{make\_overload()} for the rest of the paper

\begin{lstlisting}
template <typename... Types>
struct Overload : public Types... {
    template <typename... T>
    Overload(T&&... types) : Types{std::forward<T>(types)}... {}

    using Types::operator()...;
};
template <typename... Types>
auto make_overload(Types&&... instances) {
    return Overload<std::decay_t<Types>...>{std::forward<Types>(instances)...};
}
\end{lstlisting}

Now this can be used like so to generate a functor with overloaded
\texttt{operator()} methods from anonymous lambdas

\begin{lstlisting}
namespace {
    auto one = [](int) {};
    auto two = [](char) {};
    auto overloaded = make_overload(one, two);
} // namespace <anonymous>
\end{lstlisting}

In such a situation the consequences of this proposal must be considered.  The
easiest way to understand this proposal is to consider the rough syntactic
sugar that this provides.  A polymorphic lambda with a structured binding
declaration translates to a simple functor with a templated
\texttt{operator()} method with the structured binding decomposition
happening inside the function

\begin{lstlisting}
auto lambda = [](const auto [key, value]) { ... };

/**
 * Expansion of the above lambda
 */
struct ANONYMOUS_LAMBDA {
    template <std::decomposable<2> __Type>
    auto operator()(const __Type __instance) const {
        auto&& [key, value] = std::move(__instance);
        ...
    }
};
\end{lstlisting}

The \texttt{std::move()} is added to force a conversion to xvalue type because
the expression \texttt{e} (see \texttt{[dcl.struct.bind]p1}) is not an lvalue
in the structured binding declaration, and when \texttt{e} is not an lvalue,
the introduced bindings are decomposed as if \texttt{e} was an xvalue.  If
\texttt{e} was an lvalue, (i.e.  if \texttt{\&} was used as the
\texttt{ref-operator} or if \texttt{\&\&} was used and an lvalue was passed in
to the lambda) then the \texttt{std::move()} will be omitted (see the next
expansion for an example where \texttt{std::move()} is omitted)

Similarly a lambda that has two seperate groups of structured binding
declarations will translate with the decompositions happening serially within
the function body in order of binding declarations from left to right

\begin{lstlisting}
auto lambda = [](const auto [key, value], auto& [one, two, three]) { ... };

/**
 * Expansion of the above lambda
 */
struct ANONYMOUS_LAMBDA {
    template <std::decomposable<2> __One, std::decomposable<3> __Two>
    auto operator()(const One __one, Two& __two) {
        auto&& [key, value] = std::move(__one);
        auto& [one, two, three] = __two;
        ...
    }
};
\end{lstlisting}

Given the above expansions, a polymorphic lambda behaves almost identically to a
lambda with a \textt{auto} parameter type with the difference that these are
constrained to work only with parameters that are \texttt{x-decomposable}.
And nothing special happens when overloading

\begin{lstlisting}
namespace {
    auto one = [](int) {};
    auto two = [](auto [key, value]) {};
    auto overloaded = make_overload(one, two);
} // namespace <anonymous>

int main() {
    auto integer = int{1};
    auto pair = std::make_pair(1, 2);
    auto error = double{1};

    // calls the lambda named "one"
    overloaded(integer);
    // calls the lamdba named "two"
    overloaded(pair);
    // error
    overloaded(error);
}
\end{lstlisting}

\subsection{Viable orthogonal overloads}
One key point to consider here is that the concept based constraints on such
lambdas allows for the following two orthogonal overloads to work nicely with
each other

\begin{lstlisting}
namespace {
    auto lambda_one = [](auto [one, two]) {};
    auto lambda_two = [](auto [one, two, three]) {};
    auto overloaded = make_overload(lambda_one, lambda_two);
} // namespace <anonymous>

int main() {
    auto tup_one = std::make_tuple(1, 2);
    auto tup_two = std::make_tuple(1, 2, 2);
    overloaded(tup_one);
    overloaded(tup_two);

    return 0;
}
\end{lstlisting}

Since here either one lambda can be called or both, in no case can both
satisfy the requirements set forth by the compiler concept
\texttt{std::decomposable<x>}

\subsection{Access control and decompositions}
Another key point to consider is acess control within the expansion of the
lambda.  Decompositions will share the access control powers of the code in
the surrounding scope where the lambda is defined.  So if the decomposition
was in the body of the lambda and was valid (for example, even if the type
being decomposed has private \texttt{get<>()} methods)  the lambda would be
able to decompose it successfully.  For example the following code is valid

\begin{lstlisting}
class Something {
public:
    static auto make_decomposer();
private:
    std::tuple<int, int> tup{1, 2};
    template <std::size_t Index>
    int get();
};

namespace std {
template <>
class tuple_size<Something> : public std::integral_constant<std::size_t, 2> {};
template <std::size_t Index>
class tuple_element<Index, Something> {
public:
    using type = int;
};
} // namespace std

template <std::size_t Index>
int Something::get() {
    return std::get<Index>(this->tup);
}

auto Something::make_decomposer() {
    // decomposition of Something instances is allowed access to the privates
    // of Something since it's defined in a context where private members are
    // visible
    return [](auto [one, two]) {
        assert(one == 1);
        assert(two == 2);
    };
}

void foo() {
    auto something = Something{};
    auto decomposer = Something::make_decomposer();

    // the decomposition here happens in the scope of the lambda so is valid
    decomposer(something);
}
\end{lstlisting}


\section{Conversions to function pointers}
A capture-less polymorphic lambda with structured binding parameters can also
be converted to a plain function pointer.  Just like a regular polymorphic
lambda

\begin{lstlisting}
using FPtr_t = void (*) (std::tuple<int, int>);
auto f_ptr = static_cast<FPtr_t>([](auto [a, b]){});
\end{lstlisting}

So another conversion operator needs to be added to the expansion of the
polymorphic lambda with structured bindings above

\begin{lstlisting}
auto lambda = [](const auto [one, two]) { ... };

/**
 * Expansion of the above lambda in C++17 form with respect to overloading
 */
struct ANONYMOUS_LAMBDA {
    template <std::decomposable<2> __Type>
    auto operator()(const __Type __instance) const {
        auto&& [one, two] = std::move(__instance);
        ...
    }

private:
    /**
     * Cannot use operator() because that will either cause moves or copies,
     * elision isn't guaranteed to happen to function parameters (even in
     * return values)
     */
    template <std::decomposable<2> __Type>
    static auto __invoke(const __Type __instance) {
        auto&& [key, value] = std::move(__instance);
        ...
    }

public:
    /**
     * Enforce the decomposable requirement on the argument of the function
     */
    template <typename Return, std::decomposable<2> Arg>
    operator Return(*)(Arg)() const {
        return &__invoke;
    }
};
\end{lstlisting}

And like regular polymorphic lambdas, returning the address of the static
function invokes an instantiation of the function with the types used in the
conversion operator

\section{Exceptions}
Any exceptions during copy/move construction of the instance which is to be
decomposed being will be thrown from the call site, just as with regular
polymorphic lambdas.  However, if there is an exception thrown during the
decomposition process, for example if \texttt{get<>()} throws, that will
propagate from within the lambda.  So if function level try catch blocks were
allowed for lambdas, those would catch any exceptions generated during the
decomposition process, whereas exceptions from the copy/move construction for
the creation of the entity \texttt{e} (see \texttt{[dcl.struct.bind]p1}) would
not be caught by the imaginary function level try-catch block.


\section{Acknowledgements}

Thanks to Eric Niebler and Nicol Bolas for the suggestions and the help with
this paper!


\section{Changes to the current C++17 standard}
\subsection{Section 8.1.5.1 (\texttt{[expr.prim.lambda.closure]}) paragraph 3}

For a generic lambda, the closure type has a public inline function call
operator member template (17.5.2) whose \texttt{\justify template-parameter-list}
consists of one invented type \texttt{template-parameter} for each occurrence
of \texttt{auto} in the lambda’s \texttt{parameter-declaration-clause}, in
order of appearance.  \hlgreen{For each occurrence of a structured binding with
cardinality \texttt{x}, the \texttt{template-parameter-list} consists of an
invented type \texttt{template-parameter} with the constraint that it has to
be decomposable into \texttt{x} structured bindings (see
\texttt{[dcl.struct.bind]}).  And as such the function template only
participates in overloading when all the structured bindings are appropriately
decomposable.} The invented type \texttt{template-parameter} is a parameter
pack if the corresponding \texttt{parameter-declaration} declares a function
parameter pack (11.3.5).  \hlgreen{A structured binding
\texttt{parameter-declaration} cannot be used to invent a parameter pack.} The
return type and function parameters of the function call operator template are
derived from the \texttt{lambda-expression}’s \texttt{trailing-return-type}
and \texttt{parameter-declaration-clause} by replacing each occurrence of auto
in the decl-specifiers of the \texttt{parameter-declaration-clause} with the
name of the corresponding invented \texttt{template-parameter}.


\subsection{Section 11.5 (\texttt{[dcl.struct.bind]})}

\hlgreen{5.  If the number of structured bindings introduced by a structured
binding declaration is \texttt{x}, then a type \texttt{T} is called
\texttt{x-decomposable} if the following is well formed in the least
restrictive member access scope (where privates are accessible)}
\begin{lstlisting}
auto&& [one, two, three, ... , x] = o;
\end{lstlisting}

\hlgreen{where \texttt{T} is \texttt{decltype(o)}}

\subsection{Section 23.15.4.3 Type properties (\texttt{[meta.unary.prop]})}

\begin{lstlisting}
template <typename T, std::size_t X>
struct is_decomposable;
\end{lstlisting}

\hlgreen{\textbf{Condition} The type \texttt{T} has to be
\texttt{x-decomposable} (see \texttt{[dcl.struct.bind]}p5)}

\end{document}
