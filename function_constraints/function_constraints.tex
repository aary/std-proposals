% ********************************* HEADERS ***********************************
\documentclass{article}
\usepackage[top=.75in, bottom=.75in, left=.50in,right=.50in]{geometry}
\usepackage{fancyhdr}
\usepackage{titling}
\pagestyle{fancy}
\lhead{Tightening the constraints on \texttt{std::function}}
\rhead{\thepage}
\usepackage{color}
\usepackage{graphicx}
\usepackage{listings}
\usepackage[utf8]{inputenc}
\usepackage[spanish]{babel}
\usepackage{csquotes}
\lstset{
  % language=C++,
  showstringspaces=false,
  basicstyle={\small\ttfamily},
  numberstyle=\tiny\color{gray},
  keywordstyle=\color{blue},
  commentstyle=\color{dkgreen},
  stringstyle=\color{dkgreen},
}
\usepackage[colorlinks,urlcolor={blue}]{hyperref}
%\setlength{\parskip}{1em}
\usepackage{parskip}
\usepackage{indentfirst}
\setlength{\droptitle}{-4em}
\usepackage{soul}
\usepackage{color}
\DeclareRobustCommand{\hlgreen}[1]{{\sethlcolor{green}\hl{#1}}}
\newcommand*\justify{%
  \fontdimen2\font=0.4em% interword space
  \fontdimen3\font=0.2em% interword stretch
  \fontdimen4\font=0.1em% interword shrink
  \fontdimen7\font=0.1em% extra space
  \hyphenchar\font=`\-% allowing hyphenation
}
% \usepackage{concmath}
% \usepackage[T1]{fontenc}
% ********************************* HEADERS ***********************************

\begin{document}
\title{\textbf{Tightening the constraints on \texttt{std::function}}}
\author{Aaryaman Sagar (aary@instagram.com)}
\date{February 7, 2018}
\maketitle

\section{Introduction}

\begin{lstlisting}
#include <iostream>
#include <functional>

const int& foo(std::function<const int&()> bar) {
    return bar();
}

int main() {
    std::cout << foo([] { return 1; }) << std::endl;
}
\end{lstlisting}

The above code should not compile, however with the current restraints on
\texttt{std::function} it does, and exhibits undefined behavior.

\section{The current rules}
The offending part of the current standard comes first from section
\texttt{[func.wrap.func]p2}
\begin{displayquote}
A callable type (23.14.2) F is Lvalue-Callable for argument types ArgTypes and
return type R if the expression \texttt{INVOKE<R>(declval<F&>(),
declval<ArgTypes>()...)}, considered as an unevaluated operand (Clause 8), is
well formed (23.14.3).
\end{displayquote}

And then the corresponding definition of \texttt{INVOKE<R>} in section
\texttt{[func.require]p2}
\begin{displayquote}
Define \texttt{INVOKE<R>(f, t1, t2, ..., tN)} as
\texttt{static\_cast<void>(INVOKE(f, t1, t2, ..., tN))} if R is cv void,
otherwise \texttt{INVOKE(f, t1, t2, ..., tN)} implicitly converted to R.
\end{displayquote}

The problem here is that an invocable with return type \texttt{R} and
compatible arguments is convertible to a function object of type
\texttt{std::function<Return(Args...)>} either via construction, assignment
and combination of thereof if given that \texttt{R} is implicitly convertible to
\texttt{Return}.  This requirement needs to be tightened to accomodate for the
common case of implicit conversion from prvalue to a reference type

\section{Prevalence}
Dangling references are a common problem in C++ and have caused severe
problems in critical codebases in the past.  In particular this bug has come
up in Facebook's codebase in the past and has caused SEVs.  The issue of
returned dangling references has since been patched in the offending spots in
our code - we protect against this by detecting the binding of references to
prvalue expressions.  The code for this can be found
\href{https://goo.gl/7uiti4}{here} and \href{https://goo.gl/Wa7NGr}{here}.

A prvalue to reference binding in a function's return type should never be
allowed at runtime.  And is even prohibited by the current C++17 standard
\texttt{[class.temporary]p6.10}
\begin{displayquote}
The lifetime of a temporary bound to the returned value in a function return
statement (9.6.3) is not extended; the temporary is destroyed at the end of
the full-expression in the return statement.
\end{displayquote}

This invalidates conversions from rvalues to references in the return
statement of a function.  And in practice, compilers like
\href{https://wandbox.org/permlink/qZI97GJdtllIVnTn}{gcc 7.2.0} and
\href{https://wandbox.org/permlink/lOCOoY0qnSJwIQ8j}{clang 5.0.0} warn against
this.

\section{Backwards Compatibility}
This will be a breaking change to the standard.  Previous code that might
compile will no longer compile.  In practice this is a good limitation to
impose.  There is no code that will run without undefined behavior if this
problem manifests at runtime

\subsection{Backwards compatible fix}
To illustrate that this problem is immediately fixable, there is a route that
does not impact backwards compatibility.

The problem is detectable at compile time.  So the library can warn the user
when this problem manifests at runtime with either an
\texttt{std::logic\_error} exception when \texttt{std::function} is invoked
from an unsafe wrapped callable or abort the program with a call to
\texttt{std::abort}.  This would carry no runtime overhead in the correct use
cases as the incorrect path of execution would be handled through the regular
virtual dispatch mechanism within \texttt{std::function} implementations

\section{Changes to the current C++17 standard}
\subsection{Section 23.14.3 (\texttt{[func.require]}) paragraph 2}
Define \texttt{INVOKE<R>(f, t1, t2, ..., tN)} as
\texttt{static\_cast<void>(INVOKE(f, t1, t2, ..., tN))} if R is cv void,
otherwise \texttt{INVOKE(f, t1, t2, ..., tN)} implicitly converted to R.
\textcolor{red}{Define \texttt{SAFE\_INVOKE<R>(f, t1, t2, ..., tN)} as the
expression \texttt{\justify static\_cast<void>(INVOKE(f, t1, t2, ...,
tN))} if R is cv void, otherwise \texttt{INVOKE(f, t1, t2, ..., tN)}
implicitly converted to R except where \texttt{decltype(INVOKE(f, t1, t2,
..., tN))} is a type \texttt{A} where \texttt{!(std::is\_reference\_v<R> \&\&
!std::is\_reference\_v<A>)} is \texttt{true}, in which case the
\texttt{SAFE\_INVOKE} expression does not participate in overload resolution}

\subsection{Section 23.14.13.2 (\texttt{[func.wrap.func]}) paragraph 2}
A callable type (23.14.2) F is \texttt{Lvalue-Callable} for argument types
\texttt{ArgTypes} and return type \texttt{R} if the expression
\textcolor{red}{\texttt{SAFE\_INVOKE<R>(declval<F&>(),
declval<ArgTypes>()...)}}, considered as an unevaluated operand (Clause 8), is
well formed (23.14.3).

\end{document}
