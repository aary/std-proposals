% ********************************* HEADERS ***********************************
\documentclass{article}
\usepackage[top=.75in, bottom=.75in, left=.50in,right=.50in]{geometry}
\usepackage{fancyhdr}
\usepackage{titling}
\pagestyle{fancy}
\lhead{Tuple application traits}
\rhead{\thepage}
\usepackage{color}
\usepackage{graphicx}
\usepackage{listings}
\lstset{escapechar={|}}
\usepackage[utf8]{inputenc}
\usepackage[spanish]{babel}
\usepackage{csquotes}
\lstset{
  % language=C++,
  showstringspaces=false,
  basicstyle={\small\ttfamily},
  numberstyle=\tiny\color{gray},
  keywordstyle=\color{blue},
  commentstyle=\color{dkgreen},
  stringstyle=\color{dkgreen},
}
\usepackage[colorlinks,urlcolor={blue}]{hyperref}
%\setlength{\parskip}{1em}
\usepackage{parskip}
\usepackage{indentfirst}
\setlength{\droptitle}{-4em}
\usepackage{soul}
\usepackage{color}
\DeclareRobustCommand{\hlgreen}[1]{{\sethlcolor{green}\hl{#1}}}
\newcommand*\justify{%
  \fontdimen2\font=0.4em% interword space
  \fontdimen3\font=0.2em% interword stretch
  \fontdimen4\font=0.1em% interword shrink
  \fontdimen7\font=0.1em% extra space
  \hyphenchar\font=`\-% allowing hyphenation
}
\usepackage{xcolor}
\definecolor{darkgreen}{rgb}{0.0, 0.5, 0.0}
\lstdefinestyle{base}{
  basicstyle={\color{darkgreen}\small\ttfamily},
  showstringspaces=false,
  numberstyle=\tiny\color{gray},
  keywordstyle=\color{blue},
  commentstyle=\color{dkgreen},
  stringstyle=\color{dkgreen},
}
% \usepackage{concmath}
% \usepackage[T1]{fontenc}
% ********************************* HEADERS ***********************************

\begin{document}
\title{\textbf{Tuple application traits}}
\author{Aaryaman Sagar (aary@instagram.com)}
\date{October 8, 2018 \\ Document \#: p1318 \\ Library Evolution Working Group}
\maketitle

\section{Introduction}

The previous paper (p1317) proposes removal of return type deduction with
\texttt{std::apply} and adds in an explicit trait \texttt{std::apply\_result}
to serve as the return type.  Making \texttt{std::apply} a suitable candidate
for conditionally evaluated expressions that leverage SFINAE.

This paper introduces traits to complement \texttt{std::apply\_result} that
tend to come in handy.  These are now consistent with \texttt{std::invoke}
traits.

\section{Impact on the standard}
This proposal is a pure library extension.

\section{\texttt{std::is\_applicable}}
\texttt{std::is\_applicable} (and the corresponding template variable
\texttt{std::is\_applicable\_v}) is a unary type trait that inherits from
\texttt{std::true\_type} if the passed tuple type can be expanded into an
invocation of the passed function type, each with the same value
categorizations.

\subsection{Implementation}
\begin{lstlisting}
template <typename F, typename Tuple, typename = std::void_t<>>
struct is_applicable : std::false_type {};
template <typename F, typename Tuple>
struct is_applicable<F, Tuple, std::void_t<std::apply_result_t<F, Tuple>>>
    : std::true_type {};

template <typename F, typename Tuple>
constexpr auto is_applicable_v = is_applicable<F, Tuple>::value;
\end{lstlisting}

\section{\texttt{std::is\_applicable\_r}}
\texttt{std::is\_applicable\_r} (and the corresponding template variable
\texttt{std::is\_applicable\_r\_v}) is a unary type trait that inherits from
\texttt{std::true\_type} if the passed tuple type can be expanded into an
invocation of the passed function type, each with the same value
categorizations to yield a result that is converible to \texttt{R}.

\subsection{Implementation}
\begin{lstlisting}
template <typename R, typename F, typename Tuple, typename = std::void_t<>>
struct is_applicable_r : std::false_type {};
template <typename R, typename F, typename Tuple>
struct is_applicable_r<R, F, Tuple, std::void_t<std::apply_result_t<F, Tuple>>>
    : std::is_convertible<std::apply_result_t<F, Tuple>, R> {};

template <typename F, typename Tuple>
constexpr auto is_applicable_r_v = is_applicable<F, Tuple>::value;
\end{lstlisting}

\section{\texttt{std::is\_nothrow\_applicable}}
\texttt{std::is\_nothrow\_applicable} (and the corresponding template variable
\texttt{std::is\_nothrow\_applicable\_v}) is a unary type trait that inherits
from \texttt{std::true\_type} if the passed tuple type can be expanded into an
noexcept invocation of the passed function type.

\subsection{Implementation}
\begin{lstlisting}
// apply_impl is for exposition only
template <class F, class T, std::size_t... I>
constexpr auto apply_impl(F&& f, T&& t, std::index_sequence<I...>) noexcept(
    is_nothrow_invocable<F&&, decltype(std::get<I>(std::declval<T>()))...>{})
    -> invoke_result_t<F&&, decltype(std::get<I>(std::declval<T>()))...> {
  return invoke(std::forward<F>(f), std::get<I>(std::forward<T>(t))...);
}

template <typename F, typename Tuple, typename = std::void_t<>>
struct is_nothrow_applicable : std::false_type {};
template <typename F, typename Tuple>
struct is_nothrow_applicable<
    F,
    Tuple,
    std::void_t<std::apply_result_t<F, Tuple>>>
    : std::bool_constant<noexcept(apply_impl(
          std::declval<F>(),
          std::declval<Tuple>(),
          std::make_index_sequence<
              std::tuple_size_v<std::decay_t<Tuple>>>{}))> {};

template <typename F, typename Tuple>
constexpr auto is_nothrow_applicable_v = is_nothrow_applicable<F, Tuple>::value;
\end{lstlisting}

\section{\texttt{std::is\_nothrow\_applicable\_r}}
\texttt{std::is\_nothrow\_applicable\_t} (and the corresponding template
variable \texttt{std::is\_nothrow\_applicable\_r\_v}) is a unary type trait
that inherits from \texttt{std::true\_type} if the passed tuple type can be
expanded into an noexcept invocation of the passed function type and yield a
result that is convertible to \texttt{R}.

\subsection{Implementation}
\begin{lstlisting}
// apply_impl is for exposition only
template <class F, class T, std::size_t... I>
constexpr auto apply_impl(F&& f, T&& t, std::index_sequence<I...>) noexcept(
    is_nothrow_invocable<F&&, decltype(std::get<I>(std::declval<T>()))...>{})
    -> invoke_result_t<F&&, decltype(std::get<I>(std::declval<T>()))...> {
  return invoke(std::forward<F>(f), std::get<I>(std::forward<T>(t))...);
}

template <typename R, typename F, typename Tuple, typename = std::void_t<>>
struct is_nothrow_applicable_r : std::false_type {};
template <typename R, typename F, typename Tuple>
struct is_nothrow_applicable_r<
    R,
    F,
    Tuple,
    std::void_t<std::apply_result_t<F, Tuple>>>
    : std::conjunction<
          std::is_convertible<std::apply_result_t<F, Tuple>, R>,
          std::is_nothrow_applicable<F, Tuple>> {};

template <typename F, typename Tuple>
constexpr auto is_nothrow_applicable_r_v =
    is_nothrow_applicable_r<F, Tuple>::value;
\end{lstlisting}

\section{Changes to the current standard}
\subsection{Section 23.15.2 (\texttt{[meta.type.synop]})}
\begin{lstlisting}
// 23.15.6, type relations
\end{lstlisting}
\begin{lstlisting}
template <class Fn, class... ArgTypes> struct is_invocable;
template <class R, class Fn, class... ArgTypes> struct is_invocable_r;

template <class Fn, class... ArgTypes> struct is_nothrow_invocable;
template <class R, class Fn, class... ArgTypes> struct is_nothrow_invocable_r;
\end{lstlisting}
\begin{lstlisting}[style=base]
template <class R, class Tuple> struct is_applicable;
template <class R, class Tuple> struct is_applicable_r;

template <class R, class Tuple> struct is_nothrow_applicable;
template <class R, class Tuple> struct is_nothrow_applicable_r;
\end{lstlisting}

\begin{lstlisting}
// 23.15.6, type relations
\end{lstlisting}
\begin{lstlisting}
template <class Fn, class... ArgTypes>
  inline constexpr bool is_invocable_v = is_invocable<Fn, ArgTypes...>::value;
template <class R, class Fn, class... ArgTypes>
  inline constexpr bool is_invocable_r_v = is_invocable_r<R, Fn, ArgTypes...>::value;
template <class Fn, class... ArgTypes>
  inline constexpr bool is_nothrow_invocable_v = is_nothrow_invocable<Fn, ArgTypes...>::value;
template <class R, class Fn, class... ArgTypes>
  inline constexpr bool is_nothrow_invocable_r_v
    = is_nothrow_invocable_r<R, Fn, ArgTypes...>::value;
\end{lstlisting}
\begin{lstlisting}[style=base]
template <class Fn, class Tuple>
  inline constexpr bool is_applicable_v = is_applicable<Fn, Tuple>::value;
template <class R, class Fn, class Tuple>
  inline constexpr bool is_applicable_r_v = is_applicable_r<R, Fn, Tuple>::value;
template <class Fn, class Tuple>
  inline constexpr bool is_nothrow_applicable_v = is_nothrow_applicable<Fn, Tuple>::value;
template <class R, class Fn, class Tuple>
  inline constexpr bool is_nothrow_applicable_r_v
    = is_nothrow_applicable_r<R, Fn, Tuple>::value;
\end{lstlisting}

\subsection{Section 23.15.6 (\texttt{[meta.rel]})}
\subsubsection{\texttt{std::is\_applicable}}
\textcolor{darkgreen}{\textbf{Template}}
\begin{lstlisting}[style=base]
template <class F, class Tuple>
struct is_applicable;
\end{lstlisting}

\textcolor{darkgreen}{\textbf{Condition}} \\
\textcolor{darkgreen}{The expression \texttt{std::apply(std::declval<Fn>(),
std::declval<Tuple>())} is well-formed when treated as an unevaluated
expression}

\textcolor{darkgreen}{\textbf{Comments}} \\
\textcolor{darkgreen}{\texttt{Fn} and \texttt{Tuple} shall be complete types,
cv void or arrays of unknown bound}

\subsubsection{\texttt{std::is\_applicable\_r}}
\textcolor{darkgreen}{\textbf{Template}}
\begin{lstlisting}[style=base]
template <class R, class F, class Tuple>
struct is_applicable_r;
\end{lstlisting}

\textcolor{darkgreen}{\textbf{Condition}} \\
\textcolor{darkgreen}{The expression \texttt{std::apply(std::declval<Fn>(),
std::declval<Tuple>())} is well-formed when treated as an unevaluated
expression and yields a result that is convertible to \texttt{R}}

\textcolor{darkgreen}{\textbf{Comments}} \\
\textcolor{darkgreen}{\texttt{R}, \texttt{Fn} and \texttt{Tuple} shall be
complete types, cv void or arrays of unknown bound}

\subsubsection{\texttt{std::is\_nothrow\_applicable}}
\textcolor{darkgreen}{\textbf{Template}}
\begin{lstlisting}[style=base]
template <class R, class F, class Tuple>
struct is_nothrow_applicable;
\end{lstlisting}

\textcolor{darkgreen}{\textbf{Condition}} \\
\textcolor{darkgreen}{\texttt{is\_applicable\_v<F, Tuple>} is true
and the expression \texttt{std::apply(std::declval<F>(),
std::declval<Tuple>())} is known not to throw any exceptions.}

\textcolor{darkgreen}{\textbf{Comments}} \\
\textcolor{darkgreen}{\texttt{Fn} and \texttt{Tuple} shall be complete types,
cv void or arrays of unknown bound}

\subsubsection{\texttt{std::is\_nothrow\_applicable\_r}}
\textcolor{darkgreen}{\textbf{Template}}
\begin{lstlisting}[style=base]
template <class R, class F, class Tuple>
struct is_nothrow_applicable_r;
\end{lstlisting}

\textcolor{darkgreen}{\textbf{Condition}} \\
\textcolor{darkgreen}{\texttt{is\_applicable\_r\_v<F, Tuple>} is true
and the expression \texttt{std::apply(std::declval<F>(),
std::declval<Tuple>())} is known not to throw any exceptions.}

\textcolor{darkgreen}{\textbf{Comments}} \\
\textcolor{darkgreen}{\texttt{R}, \texttt{Fn} and \texttt{Tuple} shall be
complete types, cv void or arrays of unknown bound}
\end{document}
