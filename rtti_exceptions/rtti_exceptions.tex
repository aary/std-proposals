% ********************************* HEADERS ***********************************
\documentclass{article}
\usepackage[top=.75in, bottom=.75in, left=.50in,right=.50in]{geometry}
\usepackage{fancyhdr}
\usepackage{titling}
\pagestyle{fancy}
\lhead{Runtime type introspection with \texttt{std::exception\_ptr}}
\rhead{\thepage}
\usepackage{color}
\usepackage{graphicx}
\usepackage{listings}
\usepackage[utf8]{inputenc}
\usepackage[spanish]{babel}
\usepackage{csquotes}
\lstset{
  % language=C++,
  showstringspaces=false,
  basicstyle={\small\ttfamily},
  numberstyle=\tiny\color{gray},
  keywordstyle=\color{blue},
  commentstyle=\color{dkgreen},
  stringstyle=\color{dkgreen},
}
\usepackage[colorlinks,urlcolor={blue}]{hyperref}
%\setlength{\parskip}{1em}
\usepackage{parskip}
\usepackage{indentfirst}
\setlength{\droptitle}{-4em}
\usepackage{soul}
\usepackage{color}
\DeclareRobustCommand{\hlgreen}[1]{{\sethlcolor{green}\hl{#1}}}
\newcommand*\justify{%
  \fontdimen2\font=0.4em% interword space
  \fontdimen3\font=0.2em% interword stretch
  \fontdimen4\font=0.1em% interword shrink
  \fontdimen7\font=0.1em% extra space
  \hyphenchar\font=`\-% allowing hyphenation
}
% \usepackage{concmath}
% \usepackage[T1]{fontenc}
% ********************************* HEADERS ***********************************

\begin{document}
\title{\textbf{Runtime type introspection with \texttt{std::exception\_ptr}}}
\author{Aaryaman Sagar (aary@instagram.com)}
\date{February 7, 2018}
\maketitle

\section{Introduction}
Exceptions manifest themselves in non-linear asynchronous programming via the
\texttt{std::exception\_ptr} handle.  Asynchronous and non-linear programming
models, however tend to not mix well with exceptions because of the limited
capabilities of \texttt{std::exception\_ptr}.  This paper proposes adding RTTI
features to \texttt{std::exception\_ptr}, which will facilitate error
introspection giving users more control of exceptions without having to go
through the overhead of stack unwinding.

\section{Prevalence}
Current interfaces that aim to generalize asynchronous I/O hook into
\texttt{std::exception\_ptr} as a point of error propagation.  This is true
for example with the current \texttt{std::future} interface.  Further with
asychronous continuations this feature is going to be more widely used and
implemented.
\begin{lstlisting}
asynchronous_io().then([](std::future<Value> future) {
    try {
        cout << "Value " << future.get() << endl;
    } catch (std::runtime_error& err) {
        cerr << err.what() << endl;
    }
});
\end{lstlisting}
This interface wherein the exception is hidden behind the discriminated
asynchronous monad is convenient but quickly degrades to bad performance
because of the repeated stack unwinding, this is especially true when
exceptions propagate through several layers of chained callbacks.  Even
outside user code, implemnetations themselves might have to put a \texttt{try}
\texttt{catch} block around the callbacks just for this purpose.

When a core implementation has such drawbacks people will look to either using
some custom form of error propagation or will come up with their own
interfaces and deem exceptions underperformant.  We should solve this problem
for them and provide a well performing primitive that is already built into
the language.  As an example Facebook's folly futures, implement
\texttt{onError} callbacks and their own \texttt{folly::exception\_wrapper} to
avoid some pathological inefficiencies with exceptions and
\texttt{std::exception\_ptr}
\begin{lstlisting}
asychronous_io().then([](Value value) {
    cout << "Value " value << endl;
}).onError(std::runtime_error& err) {
    cerr << "Value " value << endl;
});
\end{lstlisting}

\section{Current \texttt{std::exception\_ptr} implementations}
Current \texttt{std::exception\_ptr} implementations contain mechanisms to
fetch \texttt{std::type\_info} object that corresponds to the type of the
exception being pointed to.  This can be found in the \texttt{libstdc++}
implementation \href{https://goo.gl/u4oYcv}{here}
\begin{lstlisting}
const std::type_info* __cxa_exception_type() const;
\end{lstlisting}

[TODO - section about relevant portions of the itanium ABI]

\section{Proposed \texttt{std::exception\_ptr} interface}
C++17 also provided a convenient utility to generalize discriminated monadic
storage - \texttt{std::any}.  Both the implementations of
\texttt{std::exception\_ptr} and \texttt{std::any} allow fetching
\texttt{std::type\_info} objects for the underlying object or exception.
Further there is another discriminated type \texttt{std::variant} that stores
one of many types.  While these interfaces solve similar problems, they have
drastically different interfaces.  This paper will go into depth about how to
unify these interfaces and allow \texttt{std::exception\_ptr} to become a well
performing first choice primitive

\texttt{std::any} provides access to an instance of any type, this is hidden
behind a type erased interface.  This closely resembles what exceptions do,
the type of the exception is hidden behind the function until the information
is made available as a part of the stack unwinding process.
\texttt{std::exception\_ptr} should provide a method to allow fetching of the
discriminated instance
\begin{lstlisting}
class exception_ptr {
public:
    // ...

    /**
     * Return an std::any object that contains a copy of the underlying stored
     * exception
     */
    std::any any() const;
};
\end{lstlisting}

This is simple and becomes a point of reusability for two separate interfaces
that solve similar problems

\subsection{Dealing with data races}
The current standard goes into depth about how to prevent data races when
accessing the same exception object through an \texttt{std::exception\_ptr}
instance.  This is why the \texttt{any()} function on
\texttt{std::exception\_ptr} should return a copy of the underlying exception
object.  Making it safe for threads to concurrently access the same
\texttt{exception\_ptr} instance.  (Provided that copy constructor
implementations for the given objects don't introduce a race themsleves)

\subsection{Dealing with recursive exception propagation}
The interaface must not return properly when an exception propagates while
copying the underlying exception instance to prevent infinite exception
recursion.  So if a call to \texttt{std::exception\_ptr::any()} causes an
exception to be thrown from the underlying exception object, the
implementation might throw a \texttt{std::bad\_exception} object possibly
causing abnormal program termination via \texttt{std::terminate}

\section{Efficient representation}
As the current proposal has been outlined a typical \texttt{std::exception\_ptr}
class is logically equivalent to a reference counted shared pointer to type
erased discriminated storage - \texttt{std::shared\_ptr<std::any>}.  However
this is just a logical representation.  Implementations are free to strip away
any unnnecesary indirections to make serialization to and from
\texttt{std::exception\_ptr} via \texttt{std::make\_exception\_ptr} and other
\texttt{std::exception\_ptr} instances performant.

Allowing \texttt{std::exception\_ptr} instances to be aware of each other's
internals also provides the bonus that we can now translate uniformly between
different \texttt{std::exception\_ptr} instances without having to go through
the overhead of stack unwinding for RTTI extraction.  This also means that we
can now limit \texttt{std::exception\_ptr} creation to a single dynamic
storage allocation

\section{Extracting the underlying exception}
It natually follows that we need an efficient method of extracting the
underlying exception from an \texttt{exception\_ptr}
\begin{lstlisting}
auto exception = exception_ptr.extract<std::runtime_error>();
\end{lstlisting}

This is implemented as if by
\begin{lstlisting}
template <typename Exc>
std::remove_cvref_t<Exc> exception_ptr::extract() {
    try {
        std::rethrow_exception(*this);
    } catch(Exc& err) {
        return err;
    } catch(...) {
        std::rethrow_exception(*this);
    }
}
\end{lstlisting}

Where the underlying exception is copied and returned (possibly more than
once).  The rules listed in \texttt{[except.handle]} apply.  If an exception
was already propagating and an incompatible exception type template was passed
to the \texttt{extract} method, as if to trigger the default catch clause
\texttt{std::terminate()} is called

\texttt{std::exception\_ptr} would now look like this
\begin{lstlisting}
class exception_ptr {
public:
    // ...

    /**
     * Return an std::any object that contains a copy of the underlying stored
     * exception
     */
    std::any any() const;

    /**
     * Extracts a copy of the underlying stored exception, if an incompatible
     * type is passed, std::rethrow_exception(*this) is called
     */
    template <typename Exc>
    Exc extract();
};
\end{lstlisting}

\section{Visitation with \texttt{std::exception\_ptr}}
Given that we have a mechanism to extract runtime type information from an
\texttt{exception\_ptr} we should have an efficient mechanism to handle errors
without going through the overhead of stack unwinding with the same conditions
as with regular exception handling
\begin{lstlisting}
exception_ptr.handle(
    [&](std::runtime_error& exc) {
        cerr << exc.what() << endl;
    },
    [&](std::logic_error& exc) {
        cerr << exc.what() << endl;
    },
    [&](std::exception& exc) {
        cerr << exc.what() << endl;
    },
    [&](...) {
        std::terminate();
    });
\end{lstlisting}

This would need to follow the same rules as exception catching via catch
clauses.  The rules listed in \texttt{[except.handle]} apply.  Here \texttt{E}
is the type of the exception stored in the \texttt{exception\_ptr} either via
\texttt{std::make\_exception\_ptr} or via a call to
\texttt{std::current\_exception()} in the presence of exception propagation
where E is \texttt{std::remove\_cvref\_t<CE>}, \texttt{CE} being the cv-ref
qualified type of the exception in the current catch clause, or the type of
the object initially thrown.

The handle clauses must be unary functions that accept a type \texttt{E} and
return void.  Polymorphic lambdas, functors with templated \texttt{operator()}
methods or invocables accepting more than one argument (either templated or
not) do not qualify as valid arguments and the resulting program is ill formed
if those are passed.

\texttt{std::exception\_ptr} would now look like this
\begin{lstlisting}
class exception_ptr {
public:
    // ...

    /**
     * Return an std::any object that contains a copy of the underlying stored
     * exception
     */
    std::any any() const;

    /**
     * Extracts a copy of the underlying stored exception, if an incompatible
     * type is passed, std::rethrow_exception(*this) is called
     */
    template <typename Exception>
    Exception extract();

    /**
     * Handles the exception as if by the same rules as normal exception
     * handling
     *
     * The HandleClauses clauses must be unary functions that accept one
     * cv-ref qualified argument and return void
     *
     * In the case where none of the handle clauses match, the exception is
     * rethrown as if via std::rethrow_exception(*this)
     *
     * A terminal closure or function that accepts elipses may be passed to
     * override the default behavior of rethrowing the exception
     */
    template <typename... HandleClauses>
    void handle(HandleClauses&&... handle_clauses);
};
\end{lstlisting}

Given the \texttt{std::exception\_ptr::extract} method,
\texttt{std::exception\_ptr::handle} performs as if implemented as so
\begin{lstlisting}
template <typename Head, typename... Tail>
void handle(HeadClause& head, Tail&&... handle_clauses) {
    // Failure case if applicable
    if constexpr (is_elipses_arg_type<Head>) {
        static_assert(sizeof...(Tail) == 0);
        head();
    }

    // handle exceptions
    try {
        head(this->extract<extract_arg_type_t<Head>>(*this));
    } catch (...) {
        this->handle(std::forward<Tail>(tail)...);
    }
}
\end{lstlisting}

Of course implementations are highly encouraged not to cause repeated stack
unwinding, as the premise of providing such introspection was to avoid
repeated stack unwinding while still allowing multiplexing many different
error types with the same exception

\end{document}
