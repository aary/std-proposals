% ********************************* HEADERS ***********************************
\documentclass{article}
\usepackage[top=.75in, bottom=.75in, left=.50in,right=.50in]{geometry}
\usepackage{fancyhdr}
\usepackage{titling}
\pagestyle{fancy}
\lhead{Runtime type introspection with \texttt{std::exception\_ptr}}
\rhead{\thepage}
\usepackage{color}
\usepackage{graphicx}
\usepackage{listings}
\usepackage[utf8]{inputenc}
\usepackage[spanish]{babel}
\usepackage{csquotes}
\lstset{
  % language=C++,
  showstringspaces=false,
  basicstyle={\small\ttfamily},
  numberstyle=\tiny\color{gray},
  keywordstyle=\color{blue},
  commentstyle=\color{dkgreen},
  stringstyle=\color{dkgreen},
}
\usepackage[colorlinks,urlcolor={blue}]{hyperref}
%\setlength{\parskip}{1em}
\usepackage{parskip}
\usepackage{indentfirst}
\setlength{\droptitle}{-4em}
\usepackage{soul}
\usepackage{color}
\DeclareRobustCommand{\hlgreen}[1]{{\sethlcolor{green}\hl{#1}}}
\newcommand*\justify{%
  \fontdimen2\font=0.4em% interword space
  \fontdimen3\font=0.2em% interword stretch
  \fontdimen4\font=0.1em% interword shrink
  \fontdimen7\font=0.1em% extra space
  \hyphenchar\font=`\-% allowing hyphenation
}
% \usepackage{concmath}
% \usepackage[T1]{fontenc}
% ********************************* HEADERS ***********************************

\begin{document}
\title{\textbf{Runtime type introspection with \texttt{std::exception\_ptr}}}
\author{Aaryaman Sagar (aary@instagram.com)}
\date{February 7, 2018}
\maketitle

\section{Introduction}
Exceptions often manifest themselves via the type erased
\texttt{std::exception\_ptr} handle.  Many of the relevant programming models
and environments however, tend to not mix well with exceptions because of the
limited capabilities of \texttt{std::exception\_ptr}.  And many platforms have
even disabled exceptions entirely.  This paper proposes adding RTTI features
to \texttt{std::exception\_ptr}, which will facilitate error introspection
without having to go through the overhead of stack unwinding and exception
propagation.

\section{Prevalence}
Current interfaces that aim to generalize asynchronous I/O hook into
\texttt{std::exception\_ptr} as a point of error propagation.  This is true
for example with the current \texttt{std::future} interface.  Further with
asychronous continuations this feature is going to be more widely used and
implemented.
\begin{lstlisting}
asynchronous_io().then([](std::future<Value> future) {
    try {
        cout << "Value " << future.get() << endl;
    } catch (std::runtime_error& err) {
        cerr << err.what() << endl;
    }
});
\end{lstlisting}
This interface wherein the exception is hidden behind the discriminated
asynchronous monad is convenient but quickly degrades to bad performance
because of the repeated stack unwinding with \texttt{std::exception\_ptr},
this is especially true when exceptions propagate through several layers of
chained callbacks.  Even outside user code, implementations themselves might
have to put a \texttt{try} \texttt{catch} block around the callbacks just for
this purpose.

When an API has such drawbacks people look to either using some custom form of
error propagation or coming up with their own interfaces and deem exceptions
underperformant.  As an example Facebook's folly futures, implement
\texttt{onError} callbacks and their own \texttt{folly::exception\_wrapper} to
avoid some pathological inefficiencies with exceptions and
\texttt{std::exception\_ptr}
\begin{lstlisting}
asychronous_io().then([](Value value) {
    cout << "Value " value << endl;
}).onError(std::runtime_error& err) {
    cerr << "Value " value << endl;
});
\end{lstlisting}

\section{Current \texttt{std::exception\_ptr} implementations}
Current \texttt{std::exception\_ptr} implementations contain mechanisms to
fetch \texttt{std::type\_info} object that corresponds to the type of the
exception being pointed to.  This can be found in the \texttt{libstdc++}
implementation \href{https://goo.gl/u4oYcv}{here}
\begin{lstlisting}
const std::type_info* __cxa_exception_type() const;
\end{lstlisting}

As far as I understand the Microsoft Visual C++ standard library has similar
private functionality.

Given the existence of such introspection mechanisms, the first addition
described in this proposal is to make the basic \texttt{type\_info} method
public

\begin{lstlisting}
class exception_ptr {
public:
    // ...

    /**
     * Queries the exception_ptr for the type of the exception object
     * contained internally
     */
    const std::type_info& type() const noexcept;
};
\end{lstlisting}

Given the above RTTI extraction interface, it is natural to also include a way
to fetch the underlying object itself.  Without which the above would be
simply a read-only operation.

\begin{lstlisting}
class exception_ptr {
public:
    // ...

    /**
     * Queries the exception_ptr for the type of the exception object
     * contained internally
     */
    const std::type_info& type() const noexcept;

    /**
     * Returns a pointer to the contained exception object
     */
    const void* get() const noexcept;
};
\end{lstlisting}

The \texttt{get()} method above returns a \texttt{const void*} pointer instead
of a \texttt{void*} because of an existing note in the standard recommending
copying to avoid issues around data races (\texttt{§[propagation]p7})

\begin{displayquote}
[\textit{Note}: If \texttt{rethrow\_exception} rethrows the same exception
object (rather than a copy), concurrent access to that rethrown exception
object may introduce a data race.  Changes in the number of
\texttt{exception\_ptr} objects that refer to a particular exception do not
introduce a data race.  \textit{- end note}]
\end{displayquote}

Returning a \texttt{const void*} forces callers to avoid unsafe concurrent
mutations on the underlying exception object.

Existing implementations of \texttt{rethrow\_exception} however do not adhere
to this guideline and rethrow the same exception object.  Whether or not this
guideline still holds is better discussed in another paper.  For the rest of
this paper, \texttt{get()} returns a \texttt{const void*} (subject to change
based on the conclusions around possible misinterpretations and/or misguided
justifications of the above note)

\section{Interoperability with \texttt{std::any}}
C++17 also provided a convenient utility to generalize discriminated monadic
storage - \texttt{std::any}.  Both the implementations of
\texttt{std::exception\_ptr} and \texttt{std::any} allow fetching
\texttt{std::type\_info} objects for the underlying object or exception.

\texttt{std::any} provides access to an instance of any type, this is hidden
behind a type erased interface.  This closely resembles what exceptions do,
the type of the exception is hidden behind the function until the information
is made available as a part of the stack unwinding process.
\texttt{std::exception\_ptr} should provide a method to allow fetching of the
discriminated instance as a \texttt{std::any}
\begin{lstlisting}
class exception_ptr {
public:
    // ...

    /**
     * Queries the exception_ptr for the type of the exception object
     * contained internally
     */
    const std::type_info& type() const noexcept;

    /**
     * Returns a pointer to the contained exception object
     */
    const void* get() const noexcept;

    /**
     * Return an std::any object that contains a copy of the underlying stored
     * exception
     */
    std::any any() const;
};
\end{lstlisting}

This is simple and becomes a point of reusability for two separate interfaces
that solve similar problems.

\subsection{What about the note discussed in \texttt{§[propagation]}?}
Like the discussion around \texttt{get()} for now,
{std::exception\_ptr::any()} returns an instance of \texttt{std::any}
initialized with a copy of the underlying exception object.  If the conclusion
is that \texttt{exception\_ptr} should provide mutable handles, then
\texttt{std::exception\_ptr::any()} should be modified to return an instance
initialized with a pointer to the underlying exception object

\subsection{Dealing with recursive exception propagation}
The interaface must not return properly when an exception propagates while
copying the underlying exception instance to prevent infinite exception
recursion.  So if a call to \texttt{std::exception\_ptr::any()} causes an
exception to be thrown from the underlying exception object, the
implementation might throw a \texttt{std::bad\_exception} object possibly
causing abnormal program termination via \texttt{std::terminate}

\section{Efficient representation}
As the current proposal has been outlined a typical \texttt{std::exception\_ptr}
class is logically equivalent to a reference counted shared pointer to type
erased discriminated storage - \texttt{std::shared\_ptr<std::any>}.  However
this is just a logical representation.  Implementations are free to strip away
any unnnecesary indirections to make serialization to and from
\texttt{std::exception\_ptr} via \texttt{std::make\_exception\_ptr} and other
\texttt{std::exception\_ptr} instances performant.

Allowing \texttt{std::exception\_ptr} instances to be aware of each other's
internals also provides the bonus that we can now translate uniformly between
different \texttt{std::exception\_ptr} instances without having to go through
the overhead of stack unwinding for RTTI extraction.  This also means that we
can now limit \texttt{std::exception\_ptr} creation to a single dynamic
storage allocation

\section{Extracting the underlying exception}
It natually follows that we need an efficient method of extracting the
underlying exception from an \texttt{exception\_ptr}
\begin{lstlisting}
auto exception = exception_ptr.extract<std::runtime_error>();
\end{lstlisting}

This is implemented \textbf{as if} by
\begin{lstlisting}
template <typename Exc>
std::optional<std::remove_cvref_t<Exc>>
exception_ptr::extract() const {
    try {
        std::rethrow_exception(*this);
    } catch(const Exc& err) {
        return err;
    } catch(...) {
        return std::nullopt;
    }
}
\end{lstlisting}

Where the underlying exception is copied (possibly more than once) and
returned.  The rules listed in \texttt{[except.handle]} apply.

\subsection{Copies? What about the note discussed in \texttt{§[propagation]}?}

If reference handles can and should be allowed from \texttt{execption\_ptr},
the \texttt{std::exception\_ptr::extract} method and hypothetical
implementation should be appropriately modified

\begin{lstlisting}
template <typename Exc>
std::optional<add_reference_wrapper_t<Exc>> exception_ptr::extract() const {
    try {
        std::rethrow_exception(*this);
    } catch(Exc err) {
        return err;
    } catch(...) {
        return std::nullopt;
    }
}
\end{lstlisting}

Note that this does not add an additional \texttt{std::remove\_cvref<T>} to
the return type to force a reference return.  Users can be allowed to extract
references to the underlying object by explicitly specifying an
\texttt{extract} operation with a ref-qualified type.  If the type is a
refernece type then the returned optional is instantiated with a
\texttt{reference\_wrapper} indirection around the given type.  For example

\begin{lstlisting}
auto one = ptr.extract<std::runtime_error>();
auto two = ptr.extract<std::runtime_error&>();
\end{lstlisting}

\texttt{std::exception\_ptr} would now look like this
\begin{lstlisting}
class exception_ptr {
public:
    // ...

    /**
     * Queries the exception_ptr for the type of the exception object
     * contained internally
     */
    const std::type_info& type() const noexcept;

    /**
     * Returns a pointer to the contained exception object
     */
    const void* get() const noexcept;

    /**
     * Return an std::any object that contains a copy of the underlying stored
     * exception
     */
    std::any any() const;

    /**
     * Extracts a copy of the underlying stored exception, if an incompatible
     * type is passed, nullopt is returned
     */
    template <typename Exc>
    std::optional<std::remove_cvref_t<Exc>> extract() const;
};
\end{lstlisting}

\section{Visitation with \texttt{std::exception\_ptr}}
Given that we have a mechanism to extract runtime type information from an
\texttt{exception\_ptr} we should have an efficient mechanism to handle errors
without going through the overhead of stack unwinding with the same conditions
as with regular exception handling.  This should look and feel familiar to
users
\begin{lstlisting}
exception_ptr.handle(
    [&](std::runtime_error& exc) {
        cerr << exc.what() << endl;
    },
    [&](std::logic_error& exc) {
        cerr << exc.what() << endl;
    },
    [&](std::exception& exc) {
        cerr << exc.what() << endl;
    },
    [&](...) {
        std::terminate();
    });
\end{lstlisting}

This would need to follow the same rules as exception catching via catch
clauses.  The rules listed in \texttt{[except.handle]} apply.  Here \texttt{E}
is the type of the exception stored in the \texttt{exception\_ptr} either via
\texttt{std::make\_exception\_ptr} or via a call to
\texttt{std::current\_exception()} in the presence of exception propagation
where E is \texttt{std::remove\_cvref\_t<CE>}, \texttt{CE} being the cv-ref
qualified type of the exception in the current catch clause, or the type of
the object initially thrown.

The handle clauses must be unary functions that accept a type \texttt{E} and
return void.  Polymorphic lambdas, functors with templated \texttt{operator()}
methods or invocables accepting more than one argument (either templated or
not) do not qualify as valid arguments and the resulting program is ill formed
if those are passed.

If the catch-all clause is not included and none of the handlers are a good
match for the exception as determined by the rules in
\texttt{[except.handle]}, \texttt{std::terminate()} is called.

\texttt{std::exception\_ptr} would now look like this
\begin{lstlisting}
class exception_ptr {
public:
    // ...

    /**
     * Queries the exception_ptr for the type of the exception object
     * contained internally
     */
    const std::type_info& type() const noexcept;

    /**
     * Returns a pointer to the contained exception object
     */
    const void* get() const noexcept;

    /**
     * Return an std::any object that contains a copy of the underlying stored
     * exception
     */
    std::any any() const;

    /**
     * Extracts a copy of the underlying stored exception, if an incompatible
     * type is passed, nullopt is returned
     */
    template <typename Exc>
    std::optional<std::remove_cvref_t<Exc>> extract() const;

    /**
     * Handles the exception as if by the same rules as normal exception
     * handling
     *
     * The HandleClauses clauses must be unary functions that accept one
     * cv-ref qualified argument and return void
     *
     * In the case where none of the handle clauses match, std::terminate() is
     * called
     *
     * A terminal closure or function that accepts elipses may be passed to
     * override the default behavior of std::terminate() being called in the
     * default case
     */
    template <typename... HandleClauses>
    void handle(HandleClauses&&... handle_clauses) const;
};
\end{lstlisting}

Given the \texttt{std::exception\_ptr::extract} method,
\texttt{std::exception\_ptr::handle} performs \textbf{as if} implemented as so
\begin{lstlisting}
template <typename Head, typename... Tail>
void handle(HeadClause& head, Tail&&... handle_clauses) const {
    // Failure case if applicable
    if constexpr (is_elipses_arg_type<Head>) {
        static_assert(
            sizeof...(Tail) == 0,
            "Ellipses handler must be last argument");
        head();
    }

    auto exception = this->extract<extract_arg_type_t<Head>&>();
    if (exception) {
        head(*exception);
        return;
    }
    this->handle(std::forward<Tail>(tail)...);
}
\end{lstlisting}

Of course implementations are highly encouraged not to cause repeated stack
unwinding, as the premise of providing such introspection was to avoid
repeated stack unwinding while still allowing multiplexing many different
error types with the same exception

\subsection{\texttt{§[propagation]p7}?}
Whether or not the note interpretation holds as discussed in the above
sections the hypothetical implementation stays the same.  An additional copy
will be created in the situation where the note interpretation holds because
of the additional \texttt{std::remove\_cvref} on the return type of
\texttt{extract}.  If it doesn't, a reference to the underlying exception will
be passed

\end{document}
